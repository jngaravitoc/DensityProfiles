%%%%%%%%%%%%%%%%%%%%%%%%%%%%%%%%%%%%%%%%%
% Thin Sectioned Essay
% LaTeX Template
% Version 1.0 (3/8/13)
%
% This template has been downloaded from:
% http://www.LaTeXTemplates.com
%
% Original Author:
% Nicolas Diaz (nsdiaz@uc.cl) with extensive modifications by:
% Vel (vel@latextemplates.com)
%
% License:
% CC BY-NC-SA 3.0 (http://creativecommons.org/licenses/by-nc-sa/3.0/)
%
%%%%%%%%%%%%%%%%%%%%%%%%%%%%%%%%%%%%%%%%%

%----------------------------------------------------------------------------------------
%	PACKAGES AND OTHER DOCUMENT CONFIGURATIONS
%----------------------------------------------------------------------------------------

\documentclass[a4paper, 12pt]{article} % Font size (can be 10pt, 11pt or 12pt) and paper size (remove a4paper for US letter paper)
\usepackage[top=1in, bottom=1.25in, left=1in, right=1in]{geometry}
\usepackage[protrusion=true,expansion=true]{microtype} % Better typography
\usepackage{graphicx} % Required for including pictures
\usepackage{wrapfig} % Allows in-line images
\usepackage{caption}
\usepackage{amsmath}
\usepackage{mathtools}
\usepackage{float}
\usepackage{mathpazo} % Use the Palatino font
\usepackage[T1]{fontenc} % Required for accented characters
\linespread{1.05} % Change line spacing here, Palatino benefits from a slight increase by default

\makeatletter
\renewcommand\@biblabel[1]{\textbf{#1.}} % Change the square brackets for each bibliography item from '[1]' to '1.'
\renewcommand{\@listI}{\itemsep=0pt} % Reduce the space between items in the itemize and enumerate environments and the bibliography

\renewcommand{\maketitle}{ % Customize the title - do not edit title and author name here, see the TITLE block below
\begin{flushright} % Right align
{\LARGE\@title} % Increase the font size of the title

\vspace{50pt} % Some vertical space between the title and author name

{\large\@author} % Author name
\\\@date % Date

\vspace{40pt} % Some vertical space between the author block and abstract
\end{flushright}
}

%----------------------------------------------------------------------------------------
%	TITLE
%----------------------------------------------------------------------------------------

\title{\textbf{Density profiles}} % Subtitle

%\author{\textsc{Juan Nicol\'as Garavito Camargo } % Author
%\\{\textit{Departamento de F\'isica\\}
%\textit{Universidad de los Andes, Bogot\'a, Colombia}}} % Institution

\date{August, 2015} % Date

%----------------------------------------------------------------------------------------
\begin{document}

\maketitle


\section{Useful Quantities and definitions}

In this section some common quantities useful for describe the denstites profile
are defined and explained.


\subsection{Critical density of the Universe:}

\subsection{Virialization}

A dark matter halo is virialized when its in equilibrium, such
an equilibrium occurs after the dark matter have collapsed and
the force of gravity equals the \textbf{relaxtion} processes
(Binney \& Tremaine pag 380).


\textbf{How is related the virialization with the radius, since what redshift 
you can define a $r_{vir}$}

A halo can be characterized using and overdensity $\Delta{vir}$
defined as ratio of the density of a virialized halo over the critical
density of the Universe $\Delta_{vir} = \frac{\rho_{vir}}{\rho_c}$.
For a cosmolgy with ($\Omega_m + \Omega_{\Lambda} = 1$)
\begin{equation}
\Delta_{vir} = (18 \pi^2 + 82x - 39x^2)/\Omega(z) 
\end{equation}
(Bryan \& Norman 1998) it's a good approximation, here $x=\Omega(z)-1$.
For the present time ($z=0$)  $\Delta_{vir}=360$.

\verb+http://arxiv.org/pdf/astro-ph/9710107v1.pdf+ \\
\verb+http://arxiv.org/pdf/astro-ph/9601088v1.pdf+


\begin{figure}[h]
\centering
\includegraphics[scale=0.7]{deltavir.png}
\end{figure}


This overdensity is enclosed in a volume which can be charcterized
with a radius $r_{vir}$ which correspond to a $M_{vir}$ given
$\Delta_{rvir}$

\begin{equation}
\rho_{vir} = \frac{3M_{vir}}{4 \pi r_{vir}^3} = \Delta_{vir} \Omega_m \rho_{crit} 
\end{equation}

\begin{equation}
r_{vir} = \left( \frac{3M_{vir}}{4 \pi \Delta_{vir} \Omega_m \rho_{crit} } \right )^{1/3}
\end{equation}

For example for a halo of mass $M = 1 \times 10^{12}M_{\odot}$ the corresponding radius is $r_{vir}=262.4$ Kpc

\subsection{$r_{200}$ \& $M_{200}$}

\begin{equation}
M_{200} = 200 \rho_c \dfrac{4}{3} \pi r_{200}^3
\end{equation}

\begin{equation}
M_{vir} = \Delta_{vir} \Omega_m \rho_c \dfrac{4}{3} \pi r_{vir}^3
\end{equation}


Matching $rho_c$ for the above two equations we get.

\begin{equation}
\dfrac{M_{200}}{M_{vir}} = \left(  \dfrac{200}{ \Delta_{vir} \Omega_m}  \right) \left( \dfrac{r_{200}}{r_{vir}}  \right)^3
\end{equation}

Here is common to call $q = \left(  \dfrac{200}{ \Delta_{vir} \Omega_m}  \right) $ at $z=0$ $q=2.053$

\begin{equation}\label{eq:q}
\dfrac{M_{200}}{M_{vir}} = q \left( \dfrac{r_{200}}{r_{vir}}  \right)^3
\end{equation}


\section{Densities profiles}

\subsection{Plummer}

The plumer density profile is one of the simplest models which describes
a constant density near the center and falls at large radius.

\begin{equation}
\rho_P (r) = \frac{3M}{4\pi a^3} (1 + \frac{r^2}{a^2})^{-5/2}
\end{equation}

Where $a$ is call the scale length. The scale length set the length $a$ in which the mayority of the density is enclosed. Note
that if $a$ is cero the plummer potential would be exactly as the potential of a point mass.
In the other hand if $a$ goes to infty the potential is rewpresenting a very extended mass source.
In other words the scale length set up the size of the volume in which the mass $M$ is enclosed.

The enclosed mass can be derived from the density by integrating over a volume.

\begin{equation}
M_P(<r) = 4 \pi \int_0^r r'^2\frac{3M}{4\pi a^3} (1 + \frac{r'^2}{a^2})^{-5/2} dr' = \frac{3M}{a^3} \left( \frac{a^4 r^3 \sqrt{r^2/a^2 + 1}}{3(r^2 + a^2)^2}  \right)
\end{equation}

\begin{equation}
M_P(<r) = M \frac{r^3}{(a^2+r^2)^{3/2}}
\end{equation}

\begin{figure}[H]
\centering
\includegraphics[scale=0.7]{plummer_density.png}
\end{figure}

\begin{figure}[H]
\centering
\includegraphics[scale=0.7]{plummer_mass.png}
\end{figure}

\begin{figure}[H]
\centering
\includegraphics[scale=0.7]{plummer_phi.png}
\end{figure}

\begin{figure}[H]
\centering
\includegraphics[scale=0.7]{plummer_velocity.png}
\end{figure}



\subsection{Hernquist}

The Hernquist profile is derived in such a way that it follows the 

\begin{equation}
\rho_{Hernquist}(r) =  \frac{M}{2\pi} \frac{a}{r(r+a)^3}
\end{equation}

\begin{equation}
M_{Hernquist}(<r) = 2aM \int \frac{r}{(r+a)^3}dr
\end{equation}

\begin{equation}
M_{Hernquist}(<r) = M \frac{r^2}{(r+a)^2}
\end{equation}

\begin{equation}
\Phi = - \frac{GM}{r+a}
\end{equation}

\begin{equation}
v_c(r) = GM \frac{r}{(r+a)^2}
\end{equation}

\begin{figure}[H]
\centering
\includegraphics[scale=0.7]{hern_density.png}
\end{figure}

\begin{figure}[H]
\centering
\includegraphics[scale=0.7]{hern_potential.png}
\end{figure}

\begin{figure}[H]
\centering
\includegraphics[scale=0.7]{hern_mass.png}
\end{figure}

\begin{figure}[H]
\centering
\includegraphics[scale=0.7]{hern_velocity.png}
\end{figure}


\subsection{Singular Isothermal Sphere}

The Singular Isothermal Sphere (\textbf{SIS}) describes a system in which the particles follow
a Maxwellian density distribution. With this distribution and the Poisson equation the follow
density profiles could be derived.


\begin{equation}
\rho_{iso}(r) = \dfrac{\sigma ^2}{2\pi G r^2}
\end{equation}

Following the same procedure as with the previuos profiles we find $M, \Phi$  and $v_c$.

\begin{equation}
M_{iso}(<r) = \dfrac{2 \sigma r}{G}
\end{equation}

\begin{equation}
\Phi_{iso}(r) = 2 \sigma^2 ln(r)  + const.
\end{equation}

\begin{equation}\label{eq:SISv}
v_c(r) = \sqrt{2}\sigma
\end{equation}

This profile is quite different to the previous ones due to the fact that here the input is
the velocity instead of the total Mass.

\begin{figure}[H]
\centering
\includegraphics[scale=0.7]{sis_density.png}
\end{figure} 

\begin{figure}[H]
\centering
\includegraphics[scale=0.7]{sis_mass.png}
\end{figure}

\begin{figure}[H]
\centering
\includegraphics[scale=0.7]{sis_phi.png}
\end{figure}


\subsection{NFW}


\begin{equation}\label{eq:rhoNFW}
\rho_{NFW}(r) = \dfrac{M}{2\pi a^3(r/a) (1 + r/a)^2}
\end{equation}


\begin{equation}\label{eq:MNFW}
M_{NFW}(r) =  M  \left(  ln(1 + x) - \frac{x}{1 + x} \right)
\end{equation}

Where $x = r/a$, is useful to define the function $f(x)$ as:

\begin{equation}
f(x) = ln(1 + x) - \frac{x}{1 + x} 
\end{equation}

Then \ref{eq:MNFW} can be expresed as:

\begin{equation}\label{eq:M2NFW}
M_{NFW} = 4 \pi \rho_a a^3 f(x)
\end{equation}

\begin{equation}\label{PhiNFW}
\Phi_{NFW} = -4\pi G M \frac{ln(1 + r/a)}{r}
\end{equation}

\begin{equation}\label{eq:cnfwz0}
c(M_{vir}) = 9.60  \left( \frac{M_{vir}}{10^{12}h^{-1}M_{\odot}} \right)^{-0.075}
\end{equation}



\begin{equation}\label{vcNFW}
v_c(r) = \sqrt{\left(\dfrac{M(r)G}{r}\right)} = \sqrt{\left( \dfrac{2 M  \left(  ln(1 + c) - \frac{c}{1 + c} \right)}{r} \right)}
\end{equation}


\begin{figure}[H]
\centering
\includegraphics[scale=0.7]{NFW_density.png}
\end{figure}

\begin{figure}[H]
\centering
\includegraphics[scale=0.7]{NFW_mass.png}
\end{figure}

\begin{figure}[H]
\centering
\includegraphics[scale=0.7]{NFW_potential.png}
\end{figure}

\begin{figure}[H]
\centering
\includegraphics[scale=0.7]{NFW_vc.png}
\end{figure}



\section{Conversion from NFW to the Hernquist profile}

The average density of the NFW distribution can be expressed as:

\begin{equation}
\bar \rho_{NFW}(r) = \dfrac{3M_{NFW}(r)}{4 \pi r^3} 
\end{equation}

And with eq.\ref{eq:M2NFW} the $\bar{\rho_{NFW}(r)}$ takes de form:

\begin{equation}
\bar \rho_{NFW}(r) = 3 \rho_a \left( \dfrac{a}{r} \right)^{3}  f(x)
\end{equation}

Now if we want to find the relationship betwee $r_{200}$ and $r_{vir}$ 
for the NFW profile we have to apply eq\ref{eq:q}. 

\begin{equation}
q = \dfrac{3 \rho_a \dfrac{a}{r_{200}} f(c_{200})}{3 \rho_a \dfrac{a}{r_{vir}}f(c)} = \dfrac{c_{200}^{3}f(c_{200})}{c_{vir^3}f(c_{vir})}
\end{equation}


\begin{equation}\ref{eq:c200cvir}
\dfrac{c_{200}}{c_{vir}} = \left( \dfrac{f(c_{200})}{qf(c_{vir})} \right)^{1/3}
\end{equation}

For $c_{vir} = 10$ this function is shown in Fig.\ref{fig:c200cvir}, where 
$y = \dfrac{c_{200}}{c_{vir}} - \left( \dfrac{f(c_{200})}{qf(c_{vir})} \right)^{1/3}$

\begin{figure}[H]\label{fig:c200cvir}
\centering
\includegraphics[scale=0.7]{c200cvir.png}
\end{figure}

Note that the solution of Eq.\ref{eq:c200cvir} is when $y=0$, one 
solution is $c_{200}=0$ but this is not of particular interest for us. 

The other solution is computed analytically using the bisection algorithm.
$c_{200} = 7.4$



\section{Miyamoto-Nagai Disk}

\section{TO-DO:}

\begin{itemize}

\item Study the properties and different parameters ($M, /pho, v_c, a$)

\item Plummer
\item Henrquist
\item Isothermal
\item NFW
\item Miyamoto-Nagai
\item Exponential
\item Jhonston and Bullock 2005 (Logarithmic)
\item Besla 2007
\item Law \& Majewusky 2010 triaxial

\item Try to reproduce MW rotation curve van Der Marel 2012 ($M_{vir}$ \& $c$ )


\item note: Klypin relation between $c$ and $M_{vir}$ Doesn't take into 
account adiabatic contraction. 

\item work in the code that integrates the orbitas using the accelerations. 

\end{itemize}

\end{document}
